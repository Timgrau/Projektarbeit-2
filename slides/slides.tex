
\documentclass[compress,aspectratio=1610]{beamer}

\usepackage[ngerman]{babel}
\usepackage[english]{isodate}
\usepackage{amssymb}
\isodate
\usepackage{multicol}
\usepackage{tikz}
\usepackage{xcolor}
\usetikzlibrary{matrix,calc}
\usepackage{minted}

%% THEMES : vertex or green %%
%% OPTIONS: vertex & green %%
%%          [bincount] o. [bincount, totalframes]
%%          [hexcount] o. [hexcount, totalframes]
%%          []         o. [totalframes]
%% OPTIONS: vertex %%
%%          [dark] 
%%          [simplefootline]

%\usetheme[simplefootline, bincount]{vertex}
%\usetheme[totalframes]{green}
\usetheme[hexcount]{vertex}

\usepackage{blkarray}
\linespread{1.1}

\title{Quantencomputer}
\subtitle{Werkzeuge f\"ur die Algorithmenimplementierung}
\date{\today}
\author{Timo Grautst\"uck}
\institute[]{
  Fachhochschule Dortmund \\
  FB 10: Informationstechnik
}
\begin{document}

\maketitle

\section*{Inhalt}
\begin{frame}
  \frametitle{Inhalt}
  \centering
  \tableofcontents[hideallsubsections]
\end{frame}

\section{Grundlagen Quantencomputer}
\begin{frame}{Warum Quantencomputer?}
  \begin{minipage}{0.6\textwidth}
    \begin{itemize}
    \item Shors Algorihmus: $\mathcal{O}((\log n)^3)$
    \item Grover Algorithmus: $\mathcal{O}(\sqrt{N})$
    \item Quantenkryptographie: key exchange problem
    \item \textbf{Quanten\"uberlegenheit \textit{(Quantum Supremacy)}}
    \end{itemize}
  \end{minipage}
  \hfill
  \begin{minipage}{0.3\textwidth}
    \begin{figure}[h]
      \centering
      \includegraphics[width=1\textwidth]{figures/Peter_Shor_2017.png}
      \caption{Peter Shor 2017}
    \end{figure}
  \end{minipage}
  \vfill
  \tiny{Bildquelle (wikimedia): \url{https://commons.wikimedia.org/wiki/File:Peter_Shor_2017_Dirac_Medal_Award_Ceremony.png}}
\end{frame}

\begin{frame}{Quanten supremacy}
  \begin{minipage}{0.45\textwidth}
    \centering
    \includegraphics[width=1.1\textwidth]{figures/Quantum-Supremacy.png}
  \end{minipage}
  \hfill
  \begin{minipage}{0.45\textwidth}
    \centering
    \includegraphics[width=0.7\textwidth]{figures/The-Future.png}
  \end{minipage}
  \vfill
  \tiny{\href{https://www.nature.com/articles/s41586-019-1666-5}{Quantum supremacy using a programmable superconducting processor} \cite{Arute2019}}
\end{frame}    


\begin{frame}{Quantenbits I}
  \begin{block}{Basiszust\"ande}
    \begin{equation}\nonumber
      |0\rangle = \begin{bmatrix}
        1 \\
        0 \\
      \end{bmatrix}
      \,\,\, \,\,\,
      |1\rangle = \begin{bmatrix}
        0 \\
        1 \\
      \end{bmatrix}
    \end{equation}
  \end{block}
  \begin{block}{Zweizustandssystem}
    Kann sich in einer Superposition der Basiszust\"ande befinden:
    \begin{equation}\nonumber
      \begin{gathered}
        |\psi\rangle = \alpha |0\rangle+\beta |1\rangle = \begin{bmatrix}
          \alpha \\
          \beta \\
        \end{bmatrix} \qquad \alpha, \beta \in \mathbf{C} \\[0.5em]
        |\psi\rangle = \cos\left(\frac{\theta}{2}\right)|0\rangle+e^{i\phi}\sin\left(\frac{\theta}{2}\right)|1\rangle \qquad \theta, \phi \in \mathbf{R}
      \end{gathered}
    \end{equation}
  \end{block}
  \begin{itemize}
  \item $\phi$ und $\theta$ sind Kugelkoordinaten, der Radius $r = 1$
  \end{itemize}
\end{frame}

\begin{frame}{Quantenbits II}
  \begin{minipage}{0.45\textwidth}
    \begin{block}{Normalisierung}
      \begin{equation}\nonumber
        \begin{gathered}
          \langle \psi | \psi \rangle = 1 \\
          \Rightarrow |\alpha|^2 +|\beta|^2 = 1
        \end{gathered}
      \end{equation}
    \end{block}
    \begin{block}{Beispiel: \color{white}\phi=0 und \theta=\pi/2}
      \begin{equation}\nonumber
        \begin{gathered}
          |\psi\rangle = \cos\left(\frac{\pi}{4}\right)|0\rangle+e^{i0}\sin\left(\frac{\pi}{4}\right)|1\rangle \\[0.5em]
          \Rightarrow |\psi\rangle = |+\rangle = \frac{1}{\sqrt{2}}\left(|0\rangle+|1\rangle\right)
        \end{gathered}
      \end{equation}
    \end{block}
  \end{minipage}
  \hfill
  \begin{minipage}{0.45\textwidth}
    \begin{figure}
      \centering
      \includegraphics[width=0.8\textwidth]{~/sciebo/Projektarbeit-2/ausarbeitung/figures/blochsphere.pdf}
      \caption{Bloch-Kugel \textit{(Blochsphere)}}
    \end{figure}
  \end{minipage}
\end{frame}

\begin{frame}{Quantenbits III}
  \begin{block}{Quantenregister}
    \begin{itemize}
    \item $n$ Qubits besitzten $2^n$ Wahrscheinlichkeitsamplituden 
    \end{itemize}
    \begin{equation}\nonumber
      |\psi \rangle = \alpha_{00} |00\rangle + \alpha_{01} |01\rangle + \alpha_{10} |10\rangle + \alpha_{11} |11\rangle = \begin{bmatrix} \alpha_{00} \\ \alpha_{01} \\ \alpha_{10} \\ \alpha_{11} \end{bmatrix}
    \end{equation}
    \begin{itemize}
    \item K\"onnen durch Produkte der Basiszust\"ande beschreiben werden
    \end{itemize}
    
    \begin{equation}\nonumber
      \begin{aligned}
        |0\rangle\otimes |0\rangle = |00\rangle = \begin{bmatrix}1 \times \begin{bmatrix} 1\\0\end{bmatrix}\\[1em] 0 \times \begin{bmatrix} 1\\0\end{bmatrix}\end{bmatrix} = \begin{bmatrix} 1 \\ 0 \\ 0 \\ 0\end{bmatrix}
      \end{aligned}
    \end{equation}
  \end{block}
\end{frame}

\begin{frame}{Quantengatter I}
  \begin{itemize}
  \item Manipulation von Qubits
  \item Quantengatter die auf $n$ Bits operieren sind unit\"are $2^n\times 2^n$-Matrizen
  \item Für diese Matrizen existieren Eigenvektoren 
  \end{itemize}
  \begin{minipage}{0.45\textwidth}
    \begin{block}{Unit\"ar}
      \begin{equation}\nonumber
        \boxed{I = A^{\dagger}A} \qquad A^{\dagger} = A^{\ast T}
      \end{equation}
    \end{block}
    \begin{block}{Eigenvektor \& Eigenwert}
      $$A|\psi\rangle = \lambda |\psi\rangle = e^{2\pi i \theta}|\psi\rangle$$
    \end{block}
  \end{minipage}
  \hfill
  \begin{minipage}{0.45\textwidth}
    \begin{block}{Beispiel: X- \& H-Gatter}
      \begin{equation}\nonumber
        \begin{aligned}
          X &= |0\rangle\langle1|+|1\rangle\langle0| =
          \begin{bmatrix}
            0 & 1 \\
            1 & 0
          \end{bmatrix}\\[1em]
          H &= |0\rangle\langle+|+|1\rangle\langle-| = \frac{1}{\sqrt{2}}
          \begin{bmatrix}
            1 & 1 \\
            1 & -1
          \end{bmatrix}
        \end{aligned}
      \end{equation}
    \end{block}
  \end{minipage}
\end{frame}

\begin{frame}{Quantengatter II - \textit{(Pauligatter)}}
  \begin{table}[h] 
    \begin{tabular}{@{\hspace{0.7cm}}c@{\hspace{0.7cm}} | @{\hspace{0.7cm}}c@{\hspace{0.7cm}} | @{\hspace{0.8cm}}c@{\hspace{0.7cm}}}
      \hline 
      Matrix & Schaltungssymbol & Wahrheitstabelle \\
      \hline & \\
      $X = \begin{bmatrix} 0 & 1 \\ 1 & 0 \end{bmatrix}$ &
      \raisebox{-.3\height}{\includegraphics[width=0.2\textwidth]{~/sciebo/Projektarbeit-2/ausarbeitung/figures/pauli_x.pdf}} &
      \begin{tabular}{|c||c||c|}
        \hline
        Fall & $|q\rangle$ & $X|q\rangle$ \\
        \hline \hline 
        1 & $|0\rangle$ & $|1\rangle$ \\
        2 & $|1\rangle$ & $|0\rangle$ \\
        \hline
      \end{tabular} \\&\\

      %%%%%%%%%%%%%%%%%%%%%%%%%%%%%%%%%%%%%%%%%%%%%%%%%%%%%%%%%%%%%%

      $Y = \begin{bmatrix} 0 & -i \\ i & 0 \end{bmatrix}$ &
      \raisebox{-.3\height}{\includegraphics[width=0.2\textwidth]{~/sciebo/Projektarbeit-2/ausarbeitung/figures/pauli_y.pdf}} &
      \begin{tabular}{|c||c||c|}
        \hline
        Fall & $|q\rangle$ & $Y|q\rangle$ \\
        \hline \hline 
        1 & $|0\rangle$ & $i|1\rangle$ \\
        2 & $|1\rangle$ & $-i|0\rangle$ \\
        \hline
      \end{tabular} \\&\\

      %%%%%%%%%%%%%%%%%%%%%%%%%%%%%%%%%%%%%%%%%%%%%%%%%%%%%%%%%%%%%%

      $Z = \begin{bmatrix} 1 & 0 \\ 0 & -1 \end{bmatrix}$ &
      \raisebox{-.3\height}{\includegraphics[width=0.2\textwidth]{~/sciebo/Projektarbeit-2/ausarbeitung/figures/pauli_z.pdf}} &
      \begin{tabular}{|c||c||c|}
        \hline
        Fall & $|q\rangle$ & $Z|q\rangle$ \\
        \hline \hline 
        1 & $|0\rangle$ & $|0\rangle$ \\
        2 & $|1\rangle$ & $-|1\rangle$ \\
        \hline
      \end{tabular} \\&\\
      \hline
    \end{tabular}
    \caption{Pauli Gatter}
  \end{table}
\end{frame}

\begin{frame}{Quantengatter III - \textit{(kontrollierte Gatter)}}
  \begin{minipage}{0.6\textwidth}
    \begin{itemize}
    \item Alle Gatter k\"onnen kontrolliert auf $n$ Qubits angewandt werden
    \item Ein Zielqubit und $n-1$ kontrollierende Qubits
    \item Um eine Transformation auf einem Zielbit auszuf\"uhren, m\"ussen sich alle kontrollierenden Bits im Zustand $|1\rangle$ befinden
    \end{itemize}
  \end{minipage}
  \hfill
  \begin{minipage}{0.3\textwidth}
    \begin{figure}
      \centering
      \includegraphics[width=0.7\textwidth]{~/sciebo/Projektarbeit-2/ausarbeitung/figures/cnot.pdf}
      \caption{Kontrolliertes-Nicht-Gatter}
    \end{figure}
  \end{minipage}
  \begin{block}{Beispiel: CNOT}
    $$CX_{01} = |0\rangle\langle0|\otimes I +|1\rangle\langle1|\otimes X =
    \begin{bmatrix}
      1 & 0 & 0 & 0 \\
      0 & 1 & 0 & 0 \\
      0 & 0 & 0 & 1 \\
      0 & 0 & 1 & 0 \\
    \end{bmatrix}$$
  \end{block}
\end{frame}

\begin{frame}{Quantengatter IV - \textit{(Toffoli-Gate)}}
  \begin{table}[h]
    \begin{tabular}{c|c|c}
      \hline 
      Matrix & Schaltungssymbol & Wahrheitstabelle \\
      \hline & \\
      %%%%%%%%%%%%%%%%%%%%%%%%%%%%%%%%%%%%%%%%%%%%%%%%%%%%%%%%%%%%%%

      $CCX_{012} = \begin{bmatrix} I_2 & 0_2 & 0_2 & 0_2 \\ 0_2 & I_2 & 0_2 & 0_2 \\ 0_2 & 0_2 & I_2 & 0_2 \\ 0_2 & 0_2 & 0_2 & X \end{bmatrix}$ &
      \raisebox{-.5\height}{\includegraphics[width=0.16\textwidth]{~/sciebo/Projektarbeit-2/ausarbeitung/figures/toffoli.pdf}} &
      \begin{tabular}{|c||c||c|}
        \hline
        Fall & $|q_0 q_1 q_2 \rangle$ & $CCX_{012}|q_0 q_1 q_2\rangle$ \\
        \hline \hline 
        1 & $|000\rangle$ & $|000\rangle$ \\
        2 & $|001\rangle$ & $|001\rangle$ \\
        3 & $|010\rangle$ & $|010\rangle$ \\
        4 & $|011\rangle$ & $|011\rangle$ \\
        5 & $|100\rangle$ & $|100\rangle$ \\
        6 & $|101\rangle$ & $|101\rangle$ \\
        7 & $|110\rangle$ & $|111\rangle$ \\
        8 & $|111\rangle$ & $|110\rangle$ \\
        \hline
      \end{tabular} \\&\\
      \hline
    \end{tabular}
    \caption{Toffoli-Gatter}
    \label{table:2Qubit-Gatter}
  \end{table}
\end{frame}

\begin{frame}{Quantenschaltungen}
  \begin{itemize}
  \item Grundlage f\"ur Quantenalgorithmen
  \item Keine R\"uckf\"uhrungen (azyklisch)
  \item Kopieren und Zusammenf\"uhren von Qubits ist nicht erlaubt
  \end{itemize}
  \begin{figure}[h]
    \centering
    \includegraphics[width=0.6\textwidth]{figures/half-adder.pdf}
    \caption{Quantenschaltung f\"ur einen Halbaddierer}
  \end{figure}
\end{frame}

\section{Werkzeuge zur Implementierung}
\begin{frame}{Werkzeuge}
  \begin{itemize}
  \item IBM Quantum Composer
    \begin{itemize}
    \item Halbaddierer bauen
    \item Hardware 
    \item Quantum Lab
    \end{itemize}
  \item Qiskit
  \end{itemize}
\end{frame}

\section{Quantenalgorithmen}
\begin{frame}{Quanten Fourier-Transformation I}
  \begin{itemize}
  \item Wird in vielen anderen Algorithmen genutzt z.B. Shor's
  \item Transformiert einen Basiszustand $|x\rangle$ zu einem Fourierzustand $|\tilde{x}\rangle$
  \item $N = 2^n$; $n$ ist die Anzahl der genutzten Qubits
  \end{itemize}
  \begin{block}{QFT}
    $$QFT \Rightarrow |\tilde{x}\rangle = \frac{1}{\sqrt{N}}\sum\limits_{j=0}^{N-1}e^{i\frac{2\pi}{N}xj}\cdot|j\rangle$$
  \end{block}
  \begin{block}{Als Produkt - \textit{(siehe \cite{Qiskit-Textbook})}}
    $$|\tilde{x}\rangle = \frac{1}{\sqrt{N}} \left(|0\rangle + e^{i\frac{2\pi x}{2^1}}|1\rangle\right)\otimes\left(|0\rangle + e^{i\frac{2\pi x}{2^2}}|1\rangle\right)\otimes\dots\otimes\left(|0\rangle + e^{i\frac{2\pi x}{2^n}}|1\rangle\right)$$
  \end{block}
\end{frame}

\begin{frame}{Quanten Fourier-Transformation II}
  \begin{block}{kontrollierte Rotation \cite{Qiskit-Textbook}}
    $$R_k =
    \begin{bmatrix}
      1 & 0 \\
      0 & e^{i\frac{2\pi}{2^k}}
    \end{bmatrix}$$
  \end{block}
  \begin{figure}[h]
    \centering
    \includegraphics[width=0.9\textwidth]{~/sciebo/Projektarbeit-2/ausarbeitung/figures/QFT.pdf}
    \caption{Schaltung der Quantum Fourier-Transformation vgl. \cite{nielsen_chuang_2010}}
  \end{figure}
\end{frame}

\begin{frame}{Quanten-Phasenschätzung I}
  \begin{minipage}{0.6\textwidth}
    \begin{itemize}
    \item Quantumteil des Shors Algorithmus
    \item Operiert auf zwei Quantumregister
    \item Nutzt Phasenr\"ucksto\ss \, \textit{(phase kickback)}, um die Phase eines unit\"aren Operators $U$ (im Fourierzustand) in ein Quantenregister zu schreiben \cite{Qiskit-Textbook}.
    \end{itemize}
  \end{minipage}
  \begin{minipage}{0.35\textwidth}
    \begin{block}{Eigenvektor}
      $$U|\psi\rangle = e^{2\pi i\theta}|\psi\rangle$$
      \begin{itemize}
      \item QPE bestimmt $\theta$
      \end{itemize}
    \end{block}
  \end{minipage}
  \begin{block}{QPE - \textit{(siehe \cite{Qiskit-Textbook})}}
    $$\frac{1}{\sqrt{2^t}}\sum\limits_{k=0}^{2^t-1}e^{2\pi i\theta k}|k\rangle = \frac{1}{\sqrt{2^t}} \left(|0\rangle+e^{2\pi i 2^{t-1}\theta}|1\rangle\right)\otimes\left(|0\rangle+e^{2\pi i 2^{t-2}\theta}|1\rangle\right)\otimes\dots\otimes\left(|0\rangle+e^{2\pi i 2^{0}\theta}|1\rangle\right)$$
  \end{block}
\end{frame}

\begin{frame}{Quanten-Phasensch\"atzung II}
  \begin{figure}
    \centering
    \includegraphics[width=0.9\textwidth]{~/sciebo/Projektarbeit-2/ausarbeitung/figures/QPE.pdf}
    \caption{Schaltung der Quanten-Phasensch\"atzung vgl. \cite{nielsen_chuang_2010}}
  \end{figure}
\end{frame}

\begin{frame}{Quanten-Phasensch\"atzung III - \textit{(Beispiel)}}
  \begin{itemize}
  \item $t = 3$ Qubits
  \end{itemize}
  \begin{block}{Gatter}
    $$U = P\left(\phi = \frac{\pi}{2}\right) =
    \begin{bmatrix}
      1 & 0 \\
      0 & e^{\frac{i\pi}{2}}
    \end{bmatrix}$$
    $$P|1\rangle =
    \begin{bmatrix}
      1 & 0 \\
      0 & e^{\frac{i\pi}{2}}
    \end{bmatrix}
    \begin{bmatrix}
      0 \\ 1
    \end{bmatrix} = e^{\frac{i\pi}{2}}|1\rangle$$
  \end{block}
  \begin{block}{Erwartung}
    $$U|\psi\rangle = e^{2\pi i\theta}|\psi\rangle$$
    \begin{itemize}
    \item $\theta = \frac{1}{4}$
    \end{itemize}
  \end{block}
\end{frame}

\setbeamertemplate{headline}{}
\section*{The End}
\begin{frame}
  \begin{center}
    \vspace{-1cm}
    \begin{tabular}{cccc}
      \includegraphics[page=3, width=0.2\textwidth]{figures/End.pdf} &
      \includegraphics[page=4, width=0.2\textwidth]{figures/End.pdf} &
      \includegraphics[page=5, width=0.2\textwidth]{figures/End.pdf} &
      \includegraphics[page=6, width=0.2\textwidth]{figures/End.pdf} \\

      \includegraphics[page=7, width=0.2\textwidth]{figures/End.pdf} &
      \includegraphics[page=8, width=0.2\textwidth]{figures/End.pdf} &
      \includegraphics[page=9, width=0.2\textwidth]{figures/End.pdf} &
      \includegraphics[page=10, width=0.2\textwidth]{figures/End.pdf} \\

      \includegraphics[page=11, width=0.2\textwidth]{figures/End.pdf} &
      \includegraphics[page=12, width=0.2\textwidth]{figures/End.pdf} &
      \includegraphics[page=13, width=0.2\textwidth]{figures/End.pdf} &
      \includegraphics[page=14, width=0.2\textwidth]{figures/End.pdf} \\

      \includegraphics[page=15, width=0.2\textwidth]{figures/End.pdf} &
      \includegraphics[page=16, width=0.2\textwidth]{figures/End.pdf} &
      \includegraphics[page=17, width=0.2\textwidth]{figures/End.pdf} &
      \includegraphics[page=18, width=0.2\textwidth]{figures/End.pdf} \\
    \end{tabular}
    \vspace{-0.2cm}
    \begin{center}
      Danke für die Aufmerksamkeit, gibt es Fragen?
    \end{center}
  \end{center}
  %\setbeamertemplate{headline}{}
\end{frame}
\setbeamertemplate{headline}{%
  \nointerlineskip%
  \usebeamerfont{headline}%
  \begin{beamercolorbox}[wd=\paperwidth,leftskip=1em,rightskip=1em,ht=2.5ex,dp=1ex]{headline}
    {%
      \setbeamercolor{math text}{parent=headline}%
      \insertframetitle%
      \ifx\insertframesubtitle\@empty%
      \else%
          {~~}\insertframesubtitle%
          \fi%
    }%
  \end{beamercolorbox}%
}
\begin{frame}[allowframebreaks]
  \frametitle{Literaturverzeichnis}
  \bibliographystyle{ieeetr}
  \bibliography{literature.bib}
\end{frame}

\end{document}

