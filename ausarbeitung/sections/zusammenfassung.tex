In dieser Arbeit wurden Werkzeuge f\"ur die Implementierung von Quantenschaltungen vorgestellt. Diese Ausarbeitung sollte unter anderem zeigen, dass die Bedienung von Qiskit oder dem Quantum Composer allein nicht ausreicht, um Quantenschaltungen bzw. Quantenalgorithmen zu implementieren. Genau wie die Kentnisse einer Hochsprache wie C, nicht direkt gew\"ahrleisten einen Algorithmus implementieren zu k\"onnen. Denn die Implementierung eines Algorithmus geht damit einher, den Algorithmus sowie dessen Aufgabe zu verstehen. Dies erfordert f\"ur klassische Algorithmen oftmals ein fundiertes Wissen in der Mathematik, sowie der Informatik.\\
Somit ist eines der wichtigsten Werkzeuge zur Implementierung von Quantenalgorithmen, die grundlegende Funktionsweisen von Quantencomputern sowie der Quantenmechanik. Einige dieser Funktionsweisen wie z.B. Quantenbits, Quantengatter, Quantenzust\"ande, Verschr\"ankungen oder unit\"are Transformationen wurden in dieser Arbeit aufgegriffen.
Eine M\"oglichkeit effizientere Laufzeiten f\"ur klasssische Algorithmen zu erhalten, k\"onnte die \"Uberf\"uhrung des Alorithmus zu einem Quantenalgorithmus sein. Diese \"Uberf\"uhrung ist nicht trivial und erfordert eine Menge Einfallsreichtum, sowie Verst\"andnis der Quantenmechanik. Daher ist es wichtig, Quantenmechanik und Quanteninformatik Studenten im Grundstudium nahezubringen \cite{Fedortchenko_2016}. Mit quelloffenen und frei verf\"ugbaren Werkzeugen wie Qiskit und Quantum Composer wird dies erm\"oglicht und die Forschung weiter vorangetrieben.
