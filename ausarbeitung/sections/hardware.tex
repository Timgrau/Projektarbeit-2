Die verschiedenen Simulationstypen von Quantencomputern klingen vielversprechend, jedoch ist auch die reale Quantenhardware heutzutage ziemlich fortgeschritten. Denn IBM Quantum hat es geschafft einen Chip zu entwickeln, der die 100 Qubit Grenze \"uberschreitet und somit auch die M\"oglichkeiten der Simulation. \\\\
Dieser Prozessor hei\ss t IBM-Eagle und ist ein 127 Qubit Quantenprozessor, bestehend aus einem Interposer, einer Verkabelungsebene, einer Resonatorebene und Qubitebene \cite{IBM-Eagle}. Der IBM-Eagle ist eine Kombination von Techniken aus vorherigen Quantenprozessoren, diese 3D-Packaging-Technik soll auch f\"ur den noch erscheinenden Condor-Prozessor mit \"uber 1.000 Qubits geeignet sein. Durch die Nutzung des heavy-hexagon Qubit-Layouts \cite{heavy-hex} aus dem vorherigen Falcon Prozessor wird das Potential f\"ur Fehlerraten reduziert. Um die Menge an Elektronik und Verkablung im Inneren des Chips zu reduzieren, wird das aus dem vorherigen Hummingbird R2 Chip bekannte Auslesemultiplexing genutzt.\\
IBM Quantum verf\"ugt \"uber 22 unterschiedliche Quantensysteme mit maximal 127 Qubits. Jedoch hat der standardm\"a\ss ige Benutzer nur Zugriff auf sieben dieser Systeme mit maximal 5 Qubits.
