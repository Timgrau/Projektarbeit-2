Um sichere Daten\"ubertragung zu gew\"ahrleisten werden h\"aufig asymmetrische Verschl\"usselungsverfahren eingesetzt. Aktuell weit verbreitete sowie praktisch eingesetzte Verfahren basieren auf harten mathematischen Problemen wie der Faktorisierung ganzer Zahlen oder dem Berechnen diskreter Logarithmen. Diese Probleme lassen sich nicht effizient durch konventionelle Algorithmen auf Digitalrechnern l\"osen. Dies gilt jedoch nicht f\"ur Quantencomputer, durch vielversprechende Quantenalgorithmen wie dem von Shors erhofft man sich diese harten Probleme in Polynomialzeit zu l\"osen. Dies ist einer der schwerwiegendsten Gr\"unde, warum Wirtschaft und Wissenschaftler quantensichere Kryptographie sowie Quantencomputer vorantreiben wollen. In dieser Arbeit sollen Grundlagen, sowie die Funktionsweise von Quantencomputern dargestellt werden. Darunter f\"allt auch die Simulation und Programmeirung von Quantenschaltungen, sowie eine Darstellung von Quantenalgorithmen die einen Bezug zur modernen Kryptographie aufweisen.
