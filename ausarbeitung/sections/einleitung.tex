Durch die Akzeptanz von Sichtweisen und Ph\"anomenen, die sich von den klassischen und bisher bekannten unterschieden, wurde uns die M\"oglichkeit geschaffen Computer und Algorithmen zu entwickeln die Leistungsst\"arker als klassische sein k\"onnen. Diese Sichtweise erm\"oglicht Informationsverarbeitung wie sie klassisch nie m\"oglich war. Quantenmechanik, der Kern der Quanteninformationsverarbeitung, Quantencomputern und der Quantenkryptographie. Der Grund zum effizienten Finden von Perioden in Funktionen, wodurch die Faktorisierung von ganzen Zahlen in $\mathcal{O}(n^3)$ gel\"ost werden kann. Dank dieser Akzeptanz konnte die Quantenmechanik immer weiter entwickelt und ausgereift werden, sodass heutzutage Quantenprozessoren mit 127 Qubits existieren und Prozessoren mit \"uber 1.000 Qubits entwickelt werden. 
