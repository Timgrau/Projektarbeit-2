Die bisher bekantesten Quantenalgorithmen bieten eine quadratische und exponentielle Beschleunigung gegen\"uber klassischen. Dies ist der Shors Algorithmus bzw. \textit{(Shors Quantum Fourier Transformation)} und der Grovers Algorithmus.\\\\
Beide Algorithmen l\"osen unterschiedliche Probleme, z.B. ist der Grover Algorithmus ein Suchalgorithmus, der zur ungeordneten Suche dient. Dieser kann genutzt werden um in gro\ss en Datenbest\"anden z.B. ein Minimum oder Maximum zu finden. Aber auch f\"ur die Suche nach Schl�sseln in Kryptosystemen wie dem Datenverschl\"usselungsstandard \textit{(DES)} bietet der Grover Algorithmus eine schnellere Laufzeit als klassische Algorithmen.\\\\
Der Shors Algorithmus kann zur Berechnung diskreter Logarithmen oder der Zerlegung ganzer Zahlen genutzt werden. Da der Shors Algorithmus dies in Polynomialzeit tut, sind asymmetrische Verschl\"usselungsverfahren wie z.B. RSA und Diffie-Hellman Schl\"usselaustausch von Quantencomputern bedroht. Aus diesem Grund versucht man Quantensichere Verfahren zu Standartisieren. Somit sollen asymmetrische Verfahren wie McElliece, Crystal-Kyber, NTRU und Saber die man als Quantensicher bezeichnet, in naher Zukunft bedrohte Verfahren ersetzen.
Gegen die Verdopplung der Schl\"usselgr\"o\ss en in symmetrischen Verschl\"usselungsverfahren ist auch der Grovers Algorithmus machtlos, bzw. ben\"otigt auch er zu viel Zeit um valide Schl\"ussel zu finden. Aus diesem Grund, ist der Shors Algorithmus f\"ur die Kryptographie und Post-Quanten-Kryptographie von gr\"o\ss erer Bedeutung.\\\\
Der Shors-Algorithmus kann Probleme wie z.B. die Faktorisierung von Fastprimzahlen in polynomieller Zeit berechnen, da dieser Algotihmus einen Quantenanteil besitzt, der zum Finden von Perioden \textit{(periodic finding)} dient. Somit ist der Shors Algorithmus in der Lage auch Probleme zu L\"osen, die in das Finden von Perioden gewandelt werden k\"onnen. Um diesen Teil des Algorithmus zu verstehen, m\"ussen zwei weitere Quantenalgorithmen verstanden werden, die Quanten-Fourier-Transformation und die Quanten-Phasensch\"atzung.
